\usepackage{fontspec} % Custom fonts for LuaTeX and XeTeX (see https://ctan.org/pkg/fontspec?lang=en)
\usepackage{pdfpages} % Include pdf pages as figures (see https://ctan.org/pkg/pdfpages?lang=en)
\usepackage{indentfirst} % Make the first line of paragraphs be indented (see https://ctan.org/pkg/indentfirst?lang=en)
\usepackage{geometry} % Adjust margins (see https://ctan.org/pkg/geometry?lang=en)
\usepackage{siunitx} % Better units and numbers (see https://ctan.org/pkg/siunitx?lang=en)
\usepackage{matlab-prettifier} % MATLAB codes (see https://ctan.org/pkg/matlab-prettifier?lang=en)
\usepackage[americaninductors]{circuitikz} % For drawing the schematics (see https://ctan.org/pkg/circuitikz?lang=en)
\usepackage{tikz} % matlab2tikz requirement (see https://ctan.org/pkg/tikz?lang=en)
\usepackage{pgfplots} % matlab2tikz requirement (see https://ctan.org/pkg/pgfplots?lang=en)
\usepackage{amsmath} % Better math presentation (see https://ctan.org/pkg/amsmath?lang=en)
\usepackage{physics} % Better math presentation (see https://ctan.org/pkg/physics?lang=en)
\usepackage{graphicx} % Insert images (see https://ctan.org/pkg/graphicx?lang=en)
\usepackage{setspace} % For line spacing (see https://ctan.org/pkg/setspace?lang=en)
\usepackage[section]{placeins} % Force figures to appear in the section in which they are declared (see https://ctan.org/pkg/placeins?lang=en)
\usepackage[sorting=none, backref=true, style=ieee]{biblatex} % Citations (see https://ctan.org/pkg/biblatex?lang=en)
\usepackage{markdown} % Use markdown in LaTeX (see https://ctan.org/pkg/markdown?lang=en)
\usepackage[inkscapepath=output/.svg]{svg} % Insert svg images (see https://ctan.org/pkg/svg?lang=en)
\usepackage{bookmark}% Linking (loads hypreref too) (see https://ctan.org/pkg/bookmark?lang=en)
\usepackage{subcaption} % For subfigures (see https://ctan.org/pkg/subcaption?lang=en)
\usepackage{float} % For H float setting (see https://ctan.org/pkg/float?lang=en)
\usepackage{zref-savepos} % For technical drawings (see below of preabmle.tex for the use case) (see https://ctan.org/pkg/zref-savepos?lang=en)
\usepackage{etoolbox} % For referencing appendix sections (see https://ctan.org/pkg/etoolbox?lang=en)

% KOMA-Script settings:
\KOMAoptions{
    fontsize=11pt,
    twoside=false,
    chapterprefix=false,
    toc=bib,
    toc=chapterentrywithdots,
    numbers=noenddot,
    captions=tableabove,
    captions=figurebelow,
    headsepline=false,
    DIV=12,
}

% Hyperref settings:
\hypersetup{
    pdftitle={\thetitle}, 
    pdfsubject={\thedescription},
    bookmarksnumbered=true,
    colorlinks=true,
    hidelinks,
    citecolor=red,
    pdfstartview=Fit,
}

% Typography:
%\setmainfont{Whatever you want}
\setsansfont{SourceSansPro}[
    Path= fonts/,
    Extension = .ttf,
    UprightFont = *-Regular,
    ItalicFont = *-Italic,
    BoldFont = *-Bold,
    BoldItalicFont = *-BoldItalic
]
\setmonofont{SourceCodePro}[
    Path= fonts/,
    Extension = .ttf,
    UprightFont = *-Regular,
    ItalicFont = *-Italic,
    BoldFont = *-Bold,
    BoldItalicFont = *-BoldItalic
]
\setstretch{1.1}
\setparsizes{17pt}{6pt}{0pt plus 1fil}

% General package settings:
\sisetup{inter-unit-product=\cdot, per-mode=symbol}
\pgfplotsset{compat=1.9} % Guarantee backwards compatibility 
\pgfplotsset{scaled y ticks=false} % To avoid scientific notation in y-axis
\pgfplotsset{scaled x ticks=false} % To avoid scientific notation in x-axis
\addbibresource{references.bib} % Citations
\renewcommand{\contentsname}{Table of Contents} % Changing the title of TOC
\renewcommand{\chapterautorefname}{Chapter} % Capitalize C in chapter in autoref
\renewcommand{\sectionautorefname}{Section} % Capitalize S in section in autoref
\renewcommand{\subsectionautorefname}{Section} % Rename subsection with Section in autoref
\def\appendixautorefname{Appendix}

% To force figures to appear in the subsection in which they are declared:
\makeatletter
\AtBeginDocument{%
    \expandafter\renewcommand\expandafter\subsection\expandafter{%
        \expandafter\@fb@secFB\subsection
    }%
}
\makeatother

% Figure size macros:
\newcommand{\SmallSize}{0.3}
\newcommand{\MediumSize}{0.4}
\newcommand{\LargeSize}{0.55}

% Define abstract environment:
\newcommand\abstractname{Abstract}
\makeatletter
\if@titlepage
  \newenvironment{abstract}{%
      \titlepage
      \null\vfil
      \@beginparpenalty\@lowpenalty
      \begin{center}%
        \bfseries \abstractname
        \@endparpenalty\@M
      \end{center}}%
     {\par\vfil\null\endtitlepage}
\else
  \newenvironment{abstract}{%
      \if@twocolumn
        \section*{\abstractname}%
      \else
        \small
        \begin{center}%
          {\bfseries \abstractname\vspace{-.5em}\vspace{\z@}}%
        \end{center}%
        \quotation
      \fi}
      {\if@twocolumn\else\endquotation\fi}
\fi
\makeatother

% External URL style:
\renewcommand\UrlFont{\color{blue}\ttfamily\footnotesize}

% MATLAB codes:
\lstset{
    style = Matlab-editor,
    basicstyle = \footnotesize\ttfamily,
    backgroundcolor=\color{gray!7},
    frame=single,
    numbers=left,
    numberstyle=\color{gray}\ttfamily\footnotesize,
    xleftmargin=25pt,
    xrightmargin=25pt,
    belowskip=10pt,
    aboveskip=10pt,
    framesep=5pt
}

% For technical drawings (see https://tex.stackexchange.com/questions/502129/fill-remaining-page-with-image):
\newcommand{\filltopageendgraphics}[2][]{
    {\centering
    \par
    \zsaveposy{top-\thepage}% Mark (baseline of) top of image
    \vfill
    \zsaveposy{bottom-\thepage}% Mark (baseline of) bottom of image
    \smash{\includegraphics[height=\dimexpr\zposy{top-\thepage}sp-\zposy{bottom-\thepage}sp\relax,#1]{#2}}%
    \par}
}

% Markdown Settings:
\markdownSetup{
    hybrid,
    underscores=false,
    cacheDir=output/.md,
    smartEllipses,
    shiftHeadings=0,
    rendererPrototypes={
        link = {
            \begin{equation}
                #4
                \label{eq:#1}
            \end{equation}
        },
        image = {
            \begin{figure}[ht!]
                \centering
                \catcode`\%=14\relax
                \catcode`\#=6\relax
                #3
                \catcode`\%=12\relax
                \catcode`\#=12\relax
                \caption{#4}
                \label{fig:#1}
            \end{figure}
        },
        headingOne=  {
            \chapter{#1}
            \label{chap:#1}
        },
        headingTwo=  {
            \section{#1}
            \label{sec:#1}
        },
        headingThree=  {
            \subsection{#1}
            \label{subsec:#1}
        },
        emphasis = {
            \catcode`\%=14\relax
            \catcode`\#=6\relax
            #1
            \catcode`\%=12\relax
            \catcode`\#=12\relax
        },
    }
}

% To make sure appendix sections render as Appendix A.3 (see https://tex.stackexchange.com/questions/149807/autoref-subsections-in-appendix)
\def\appendixautorefname{Appendix}

\makeatletter
\patchcmd{\hyper@makecurrent}{%
    \ifx\Hy@param\Hy@chapterstring
        \let\Hy@param\Hy@chapapp
    \fi
}{%
    \iftoggle{inappendix}{%true-branch
        % list the names of all sectioning counters here
        \@checkappendixparam{chapter}%
        \@checkappendixparam{section}%
        \@checkappendixparam{subsection}%
        \@checkappendixparam{subsubsection}%
        \@checkappendixparam{paragraph}%
        \@checkappendixparam{subparagraph}%
    }{}%
}{}{\errmessage{failed to patch}}

\newcommand*{\@checkappendixparam}[1]{%
    \def\@checkappendixparamtmp{#1}%
    \ifx\Hy@param\@checkappendixparamtmp
        \let\Hy@param\Hy@appendixstring
    \fi
}
\makeatletter

\newtoggle{inappendix}
\togglefalse{inappendix}

\apptocmd{\appendix}{\toggletrue{inappendix}}{}{\errmessage{failed to patch}}
