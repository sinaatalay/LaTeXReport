\markdownRendererDocumentBegin
\markdownRendererHeadingOne{Introduction}\markdownRendererInterblockSeparator
{}An \markdownRendererStrongEmphasis{absorption dynamometer}, also known as a "\markdownRendererStrongEmphasis{dyno}", is a device that measures the instantaneous rotational speed and torque of an engine, motor, or any rotating prime mover while acting as a load. A \markdownRendererStrongEmphasis{motoring dynamometer} is a device that measures the required rotational speed and torque to operate a driven device (such as a pump) while acting as a prime mover.\markdownRendererInterblockSeparator
{}This report covers an absorption dynamometer design. The absorption dynamometer is referred to as the dynamometer throughout the report.\markdownRendererInterblockSeparator
{}A dynamometer is a helpful tool to analyze a motor's characteristics before using it on real systems. Those characteristics can help to compare a motor's performance with other options accurately.\markdownRendererInterblockSeparator
{}Motor suppliers usually provide torque vs. speed, power, and current curves; however, they may not always be reliable. Moreover, the specifications of old motors may be unknown. However, with a dynamometer, the required measurements can be made.\markdownRendererInterblockSeparator
{}\markdownRendererHeadingTwo{Need}\markdownRendererInterblockSeparator
{}A dynamometer is a valuable tool for any mechanical engineering laboratory. However, the current dynamometers used in the mechanical engineering lab at Bogazici University's Kilyos Campus require further improvement with different energy-dissipating methods, and it is not suitable for small DC motors with high rotation speed and low torque. Therefore, a new dynamometer needed to be designed and constructed.\markdownRendererInterblockSeparator
{}\markdownRendererHeadingTwo{Problem Statement and General Specifications}\markdownRendererInterblockSeparator
{}A low-cost small dynamometer will be designed to use in engineering labs. The dynamometer is expected to operate for electric motors without reduction drives (high-speed, low-torque motors).\markdownRendererInterblockSeparator
{}\markdownRendererUlBeginTight
\markdownRendererUlItem The dynamometer should be able to operate at speed and power up to 20000 RPM and 100 W, respectively.\markdownRendererUlItemEnd 
\markdownRendererUlItem The dynamometer's maximum torque measurement capability for a given input speed should be specified.\markdownRendererUlItemEnd 
\markdownRendererUlEndTight \markdownRendererInterblockSeparator
{}In general, four questions are answered to design the dynamometer.\markdownRendererInterblockSeparator
{}\markdownRendererOlBegin
\markdownRendererOlItemWithNumber{1}What type of absorption unit will be used?\markdownRendererInterblockSeparator
{}Measuring power implies the dissipation of energy. Therefore, the energy output of the test motor\footnote{The motor that is being tested.} should be absorbed somehow in the dynamometer for measurement.\markdownRendererOlItemEnd 
\markdownRendererOlItemWithNumber{2}What device will be used for torque measurement?\markdownRendererInterblockSeparator
{}The torque needs to be measured.\markdownRendererOlItemEnd 
\markdownRendererOlItemWithNumber{3}What device will be used for rotation speed measurement?\markdownRendererInterblockSeparator
{}The rotation speed needs to be measured.\markdownRendererOlItemEnd 
\markdownRendererOlItemWithNumber{4}Will the speed, torque, or both will be controlled?\markdownRendererInterblockSeparator
{}Generally, dynamometers are used to learn about a motor's performance at different speeds. Thus, the user may need to control the speed, the torque, or both.\markdownRendererOlItemEnd 
\markdownRendererOlEnd \markdownRendererInterblockSeparator
{}\markdownRendererHeadingTwo{Absorption Unit Selection}\markdownRendererInterblockSeparator
{}Dynamometers operate by applying a load (braking) to the test motor. Applying a load to a rotating device means energy dissipation, $\text{Power}=\text{(Torque)}\cdot\text{(Rotation Speed)}$.\markdownRendererInterblockSeparator
{}Dynamometers accomplish braking by using various methods to absorb the energy output. For example, the energy can be absorbed with a mechanical friction brake, hydraulic brake, eddy current brake, etc. Another method is to use another DC motor, the load motor\footnote{Another DC motor that is being used as a generator in the dynamometer.}, and couple it with the test motor so that dynamic brake (see \autoref{subsec:Dynamic Brake}) can be applied and energy can be absorbed. This type of dynamometer is generally called an \markdownRendererStrongEmphasis{electric dynamometer}, and this report covers its design \cite{474Book}.\markdownRendererInterblockSeparator
{}Before the decision, the available methods are researched and discussed. Then, the decision is made considering the limitations of our time, money, technology, and design criteria. This section briefly explains some types of absorption units and the reason behind the decision.\markdownRendererInterblockSeparator
{}\markdownRendererHeadingThree{Mechanical Friction Brake}\markdownRendererInterblockSeparator
{}Some dynamometers absorb energy in a mechanical friction brake. The easiest way to build a dynamometer with this method is to use a prony brake\footnote{The prony brake is a type of friction brake invented by Gaspard de Prony in 1821 \cite{PronyHistory}.}, as shown in \autoref{fig:PronyBrake}. A prony brake absorbs the motor's energy output and measures the torque produced. Moreover, it is a very simple device to build.\markdownRendererInterblockSeparator
{}\markdownRendererImage{PronyBrake}{\includesvg[width=\MediumSize\paperwidth]{figures/PronyBrake.svg}}{\includesvg[width=\MediumSize\paperwidth]{figures/PronyBrake.svg}}{Schematic diagram of a prony brake \cite{PronyBrakeSchematic}.}\markdownRendererInterblockSeparator
{}As shown in \autoref{fig:PronyBrake}, a lever is clamped to the shaft, and the load F is measured. The load F is only carried by the frictional force between the shaft and the lever since the lever is only supported by the shaft and the load F. Then, by using statics, it can be shown that $M=Fl$. There are two primary methods to measure F, using a weight or a load cell.\markdownRendererInterblockSeparator
{}An implementation of a prony brake dynamometer is shown in \autoref{fig:PronyBrakeYoutube}.\markdownRendererInterblockSeparator
{}\markdownRendererImage{PronyBrakeYoutube}{\includegraphics[width=\SmallSize\paperwidth]{figures/PronyBrakeYoutube.png}}{\includegraphics[width=\SmallSize\paperwidth]{figures/PronyBrakeYoutube.png}}{A small prony brake dynamometer \cite{PronyBrakeYoutube}.}\markdownRendererInterblockSeparator
{}Even though a prony brake is affordable and convenient, it is inflexible and hard to control. Either the weights or the clamping force must be changed to control the load, and both are very hard to control electronically. Another disadvantage is that frictional braking causes brake pad wear, so that a regular replacement might be required.\markdownRendererInterblockSeparator
{}\markdownRendererHeadingThree{Eddy Current Brake}\markdownRendererInterblockSeparator
{}When a piece of metal moves through a magnetic field, circulating currents called eddy currents are generated in the metal. According to Lenz's law, the generated eddy currents create opposing magnetic fields, which gives rise to a repulsive force that resists the motion of the metal \cite{PhysicsBook}. Many braking systems make use of this phenomenon.\markdownRendererInterblockSeparator
{}In its simplest form, an eddy current dynamometer uses a metal disk attached to a shaft and a magnetic field of controlled strength, as shown in \autoref{fig:EddyCurrentBrake}. Eddy currents are induced in the rotating disk, which results in braking. This is an example of an eddy current brake. The strength of braking depends on the magnitude of the magnetic field. The magnetic field's magnitude can be adjusted by moving the magnet or increasing the current passing through an electromagnet. The energy from the eddy currents is absorbed in the disk and transformed into heat.\markdownRendererInterblockSeparator
{}\markdownRendererImage{EddyCurrentBrake}{\includesvg[width=\MediumSize\paperwidth]{figures/EddyCurrentBrake.svg}}{\includesvg[width=\MediumSize\paperwidth]{figures/EddyCurrentBrake.svg}}{Schematic diagram of a eddy current brake \cite{EddyCurrentBrakeSchematic}.}\markdownRendererInterblockSeparator
{}Even though eddy current braking absorbs the motor's energy, the torque measurement is more complex than the prony brake. Moreover, building a well-balanced disk to be coupled to the shaft is not easy. However, the load might be controlled electronically with an electromagnet, unlike the prony brake.\markdownRendererInterblockSeparator
{}\markdownRendererHeadingThree{Dynamic Brake}\markdownRendererInterblockSeparator
{}A dynamic brake uses an electric motor as a generator to produce an opposing torque that resists the motion. One of the most used types of dynamometers is an electric dynamometer. In an electric dynamometer, the test motor's shaft is coupled to a generator so that the test motor's energy is absorbed with electrical output from the generator. The generated electrical power can be dissipated as heat with a resistor or returned to the supply line.\markdownRendererInterblockSeparator
{}This type of dynamometer offers many advantages like\markdownRendererInterblockSeparator
{}\markdownRendererUlBeginTight
\markdownRendererUlItem A DC motor is a very accessible device and is ready to operate as a generator.\markdownRendererUlItemEnd 
\markdownRendererUlItem A dynamic braking system's lifespan is longer than a frictional braking system's since there is no wear problem.\markdownRendererUlItemEnd 
\markdownRendererUlItem Suppose a DC motor is used as a generator. In that case, the torque can be easily calculated by measuring the current flows through the generator since DC motors exhibit a linear relationship between torque and current.\markdownRendererUlItemEnd 
\markdownRendererUlItem The load can be easily changed by adjusting the resistance connected to the generator output with a potentiometer.\markdownRendererUlItemEnd 
\markdownRendererUlItem The dynamometer could be further improved to turn the absorbed energy into useful work that would otherwise be lost as heat.\markdownRendererUlItemEnd 
\markdownRendererUlItem The absorber dynamometer might be modified to operate in reverse as a motoring dynamometer.\markdownRendererUlItemEnd 
\markdownRendererUlEndTight \markdownRendererInterblockSeparator
{}The main disadvantage of dynamic braking is that the braking torque is linearly proportional to the rotation speed. Therefore, it is not possible to brake the motor at low speeds. The details will be discussed in \autoref{chap:Theory}.\markdownRendererInterblockSeparator
{}\markdownRendererHeadingThree{The Decision}\markdownRendererInterblockSeparator
{}The eddy current brake option was rejected after considering our time and budget limitations. It was the hardest one to build and maintain among our choices, and it did not offer advantages that the others could not.\markdownRendererInterblockSeparator
{}We collected more detailed information about prony brakes and dynamic brakes. Then, we found and reviewed their implementations on the internet. Ultimately, we chose the dynamic brake and constructed an electric dynamometer. Its ease of use and many advantages, such as being electronically controllable and open to further modifications, made this option more beneficial than a prony brake dynamometer.\markdownRendererDocumentEnd